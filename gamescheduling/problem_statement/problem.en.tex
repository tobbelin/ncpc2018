\problemname{Game Scheduling}

\illustration{0.37}{chesstournament}{\href{https://commons.wikimedia.org/wiki/File:British_Chess_Championship_2009.jpg}{Picture} by Pat Paker via Wikimedia Commons, CC BY}%
\noindent
In a tournament with $m$ teams, each team consisting of $n$ players,
construct a playing schedule so that each player is paired up against all players in all teams except their own. That is, each player should play $(m-1) \cdot n$ games.

The playing schedule should be divided into rounds. A player can play at most one game per round. If a player does not play a game in a round, that player is said to have a bye in that round.

Your task is to write a program that constructs a playing schedule so that no player has a bye in more than $1$ round. In other words, the total number of rounds in the playing schedule should be no more than $(m-1) \cdot n  + 1$.

The order of the rounds and games, and who is home and away in a game, does not matter.

\section*{Input}

The input consists of a single line with two integers $n$ and $m$ ($1 \le n \le 25$, $2 \le m \le 25$, $n \cdot m \le 100$), the number of players in a team and the total number of teams, respectively.

\section*{Output}

Output one line per round in the playing schedule. Each line should contain a space separated list of games. A game is in the format ``\texttt{<player>-<player>}''. The players in the first team are denoted as $\texttt{A}1, \texttt{A}2, ..., \texttt{A}n$; the second team $\texttt{B}1, \texttt{B}2, \ldots \texttt{B}n$ and so on.
