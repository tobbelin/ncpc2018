\problemname{Code Cleanups}
\illustration{.40}{garbage}{\href{http://www.osnews.com/story/19266/WTFs_m}{Picture} by Thom Holwerda via OSnews}%
\noindent
The management of the software company JunkCode has recently found,
much to their surprise and disappointment, that productivity has gone
down since they implemented their enhanced set of coding
guidelines. The idea was that all developers should make sure that
every code change they push to the master branch of their software
repository strictly follows the coding guidelines. After all, one of the developers,
Perikles, has been doing this since long before these regulations became effective so how hard could it be?

Rather than investing a lot of time figuring out why this degradation in
productivity occurred, the line manager suggests that
they loosen their requirement:
developers can push code that weakly violates the
guidelines as long as they run cleanup phases on the code from time to
time to make sure the repository is tidy.
%She makes the argument that
%detecting if the coding guidelines are met can be
%automatically checked, and she volunteers to install a system that
%measures how much each developer's code deviates from the guidelines.

She suggests a metric where the ``dirtiness'' of a developer's code is the
sum of the pushes that violate the guidelines -- so-called \emph{dirty} pushes -- made by that developer, each weighted by the number of  days since it was pushed. The number of days since a dirty push is a step function that increases by one each midnight following the push.
Hence, if a developer has made dirty pushes on days $1$, $2$, and $5$, the
dirtiness on day $6$ is $5+4+1=10$. She suggests that a cleanup phase,
completely fixing all violations of the coding guidelines,
must be completed before the dirtiness reaches $20$.
One of the developers, Petra, senses that this rule must be obeyed not only because it is a company policy. Breaking it will also result in awkward meetings with a lot of concerned managers who all want to know why she cannot be more like Perikles? Still, she wants to run the cleanup phase as seldomly as possible,
and always postpones it until it is absolutely necessary. A cleanup phase is always run at the end of the day and fixes every
dirty push done up to and including that day.
Since all developers are shuffled to new projects at the start
of each year, no dirtiness should be left after midnight at the end of new year's eve.

\section*{Input}
The first line of input contains an integer $n$ ($1 \leq n \leq 365$),
the number of dirty pushes made by Petra during a year. The second line
contains $n$ integers $d_1, d_2, \ldots, d_n$ ($1 \leq d_i \leq 365$
for each $1 \le i \le n$) giving the days when Petra
made dirty pushes. You can assume that $d_i < d_j$ for $i < j$.


\section*{Output}
Output the total number of cleanup phases needed for Petra to keep the dirtiness strictly below $20$ at
all times.
